
%\section{Lessons, Challenges, and Future Directions}
%\label{sec:future}
%\vspace{10pt}
%
%
%~\bu{Points to expand: 
%\begin{itemize}
%\item Our work provides preliminary evidence for our hypothesis. But more work would be needed %to make it practically useful.
%\item The holy grail would be PMaaS whereby ... this would require cooperation between the %cloud provider and the tenants in the form of ...
%\item what about the costs of PMaaS itself? 
%\end{itemize}
%}

\section{Conclusions}
\label{sec:conclus}
\vspace{10pt}

The diversity of resource types offered by public clouds is much higher than in conventional privately-owned data centers. The complexity of options available makes decision on resource acquisition more complex, with the larger range of service available. Tenants may need to calibrate their application performance models for a large number of resource configurations. In a private cloud, the system on which such calibration is done is the same as the machines on which the application eventually runs. This paper investigates the possibility of exploiting this diversity to ease the tenant's performance modeling.  

To explore this idea we applied a linear regression model to the relationship between latency and throughput for several popular database servers. The model expressed the average response time of a class of interactive server applications as a linear combination of the offered throughput (requests/s), the number of CPU cores and memory in the procured VM, and degree of replication employed by the application. For three different real-world applications - Redis, Apache Cassandra, and MySQL -  the model accuracy increased for more diverse sets of VMs. 
For example, the $R^2_{predicted}$ measure of model efficacy for Redis improved from 0.4-0.5 with 2 VM types for training and 0.7 for 3 VM types to 0.8 for 4 VM types. Qualitatively similar results were observed for Apache Cassandra and MySQL. 
Although this modeling approach is very specific, the basic observation could be expanded to more accurately model more VM types in more detail.  This would lead to more interesting research challenges, such as automating the process of calibrating performance models using diverse resource types on a public cloud allowing providers to offer ``performance modeling as a service'' to their tenants. 

\begin{comment}
~\bu{List of pending things:
\begin{itemize}
\item Minor fixes to Fig. 2
\item Explain the pros and cons of choosing R-2, explain implications of num cores and memory not being independent, other modeling concerns
\item Explain the idea behind performance modeling as a service more clearly
\item go over the list of PC members and ensure their papers are cited!
\end{itemize}
}
\end{comment}


